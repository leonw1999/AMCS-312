\subsection*{(a)}
We assume the setting to be as in the homework sheet and the latencies as in the slides. Let us assume that we desire to wash $n$ loads of laundry. Without parallelization, the time to complete this task is given by
\[
T_1 = n \cdot (30 + 40 + 20) \; \text{minutes} \; = 90n \; \text{minutes}.
\]
The next student always starts drying when the previous student finished drying. Hence, under optimal execution, $n$ students solve the task in
\[
T_n = 30 + n \cdot 40 + 20 \; \text{minutes}.
\]
Thus, the speedup is given as follows and grows asymptotically: 
\[
\frac{T_1}{T_n} =\frac{9n}{3 + 4n + 2} \rightarrow \frac{9}{4} = 2.25 \; \text{ as } n \to \infty.
\]
Independent of $n$, this is executed on three processors (washing, drying, folding). Hence, the parallel efficiency is given by,
\[
E = \frac{T_1}{3 T_n} = \frac{3n}{3 + 4n + 2} \rightarrow 0.75 \; \text{ as } \; n \to \infty.
\]
The maximum parallel efficiency is thus (asymptotically) $0.75$.

\subsection*{(b)}
\FloatBarrier
We obtain the following values for the parallel efficiency:

\begin{center}
\begin{tabular}{c|c|c|c|c|c|c|c|c|c|c|c|c|}
No. of students & 1 & 2 & 3 & 4 & 5 & 6 & 7 & 8 & 9 & 10 & 11 & 12 \\ 
\hline 
Parallel efficiency & $0.333$ & $0.461$ & $0.529$ & $0.571$ & $0.6$ & $0.62$ & $0.636$ & $0.648$ & $0.658$ & $0.666$ & $0.673$ & $0.679$ \\ 
\hline 
\end{tabular}
\end{center}

With twelve students we are in the $10\%$ range, i.e. $\frac{0.75 - 0.679}{0.75} \leq 0.1$.

