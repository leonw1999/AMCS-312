\subsection*{(a)}
The Mandates are as follows:
\begin{enumerate}
\item Better resolve a model's full, natural range of length or time scales
\item Accommodate physical effects with greater fidelity
\item Allow the model degrees of freedom in all relevant dimensions
\item Better isolate artificial boundary conditions
\item Combine multiple complex models
\item Solve an inverse problem or perform data assimilation
\item Perform otimization or control
\item Quantify uncertainty
\item Accomplish predictions without physical models using statistical models based on large data sets
\end{enumerate}

\subsection*{(b)}
Accomodating physical effects with greater fidelity is an important opportunity in aerodynamic modeling for e.g. aircraft design. This includes the modeling of turbulent flows and shock waves. Still, many costly experiments need to be carried out in the wind tunnel including people and material as their virtual numerical pendants in computational fluid dynamics are not feasible. This includes especially numerical simulations of high-Reynolds-number flows. Also, wind tunnel experiments are physically not optimal to see all possible effects on a moving airplane as even here many approximations need to be made to translate the results into a real life situation. One of the potentials of HPC in this area is that given enough computation power, simulations might become more accurate than wind tunnel experiments. This raises a second important point as it will not only safe costs in the production, but it will make airplanes safer and more efficient in practice.