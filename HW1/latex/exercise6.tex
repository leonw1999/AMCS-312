\subsection*{(a)}
The Top500 HPL List is ranked by the High-Performance Linpack (HPL) Benchmark. The latter uses the LU decomposition algorithm with partial pivoting to solve a dense system of linear equations. The associated matrix has randomly distributed elements between $-1$ and $1$. The performance is measured in flop/s and the highest sustained number of flops/s achievable during this computation determines a supercomputer's ranking in the list.

The Top124 list is ranked by the High Performance Conjugate Gradient (HPCG) benchmark. the latter uses the conjugate gradient algorithm to solve large, sparse, symmetric, positive-definite linear systems. Performance is measured in flop/s and the highest sustained number of flops achievable during this computation determines a supercomputer's ranking in the list.

\subsection*{(b)}

\begin{tabular}{|c|c|c|c|c|c|c|}
Sum & HPL list & Green list & Manufacturer & Model & Nickname & Country \\ 
\hline 
19 & 1 & 18 & HPE & HPE Cray EX255a & El Capitan & USA \\ 
\hline 
21 & 7 & 14 & HPE & HPE Cray EX254n & Alps & Switzerland \\ 
\hline 
22 & 10 & 12 & HPE & HPE Cray EX255a & Tuolumne & USA \\ 
\hline 
24 & 18 & 6 & ParTec/EVIDEN & BallSequana XH3000 & JETI & Germany \\ 
\hline 
24 & 2 & 22 & HPE & HPE Cray EX235a & Frontier & USA \\ 
\hline 
26 & 5 & 21 & HPE & HPE Cray EX235a & HPC6 & Italy \\ 
\hline 
29 & 13 & 16 & HPE & HPE Cray Ex254n & Venado & USA \\ 
\hline 
33 & 20 & 13 & HPE & HPE Cray Ex255a & El Dorado & USA \\ 
\hline 
33 & 8 & 25 & HPE & HPE Cray Ex235a & LUMI & Finland \\ 
\hline 
49 & 30 & 19 & HPE & HPE Cray EX235a & Adastra & France \\ 
\hline 
\end{tabular} 


