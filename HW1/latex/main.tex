\documentclass[
a4paper,
]{article}
\usepackage{lmodern}
\usepackage[utf8]{inputenc}
\usepackage[T1]{fontenc}


\addtolength\jot{1ex}


\usepackage[utf8]{inputenc}
\usepackage{amsmath}
\usepackage{amsthm}
\usepackage{listings}
\usepackage{color}
%\usepackage{german}
%\usepackage{capt-of}
\usepackage[font=small,
figurewithin=section,labelsep=space]{caption}
\usepackage{amssymb}
\usepackage{tabto}
\usepackage{graphicx}
\usepackage{enumerate}
\usepackage{cite}
\usepackage{tikz-cd}
\usepackage{faktor}
\usepackage{verbatim}
\usepackage{dsfont}
\usepackage{svg}
\usepackage{longtable}
\usepackage{bbm}
\usepackage{mathtools}
%\usepackage[vcentermath]{youngtab}
\usepackage[left=3cm, top=2cm, bottom=2.5cm, right=3cm]{geometry}
\usepackage{appendix}
\usepackage[english]{hyperref}
\usepackage{cleveref}
\usepackage{placeins}
\usepackage{bm}
\usepackage{array, diagbox}
\setcounter{MaxMatrixCols}{20}

\let\div\relax
\DeclareMathOperator{\div}{div}
\newcommand{\dd}[1]{\frac{\partial}{\partial #1}}
\newcommand{\ddi}[2]{\frac{\partial #1}{\partial #2}}
\newcommand{\ddtwo}[1]{\frac{\partial^2}{\partial #1^2}}
\newcommand{\ddtwoi}[2]{\frac{\partial^2 #1}{\partial #2^2}}
\newcommand{\ddtwod}[3]{\frac{\partial^2 #1}{\partial #2 \partial #3}}
\DeclareMathOperator{\tr}{Tr}
\newcommand{\kl}{\left(}
\newcommand{\kr}{\right)}
\newcommand{\el}{\left[}
\newcommand{\er}{\right]}
\newcommand{\R}{\mathbb{R}}
\newcommand{\C}{\mathbb{C}}
\newcommand{\Z}{\mathbb{Z}}
\newcommand{\N}{\mathbb{N}}
\newcommand{\cl}{\mathcal{L}}
\newcommand{\cf}{\mathcal{F}}
\newcommand{\cb}{\mathcal{B}}
\newcommand{\Ocal}{\mathcal{O}}
\newcommand{\simd}{\overset{\text{d}}{\sim}}
\newcommand{\conp}{\xrightarrow{P}}
\newcommand{\cond}{\xrightarrow{\text{d}}}
\newcommand{\conas}{\xrightarrow{\text{f.s.}}}
\newcommand{\conlp}{\xrightarrow{L^p}}
\newcommand{\conlpx}[1]{\xrightarrow{L^{#1}}}
\newcommand{\conv}{\xrightarrow{\text{v}}}
\newcommand{\E}{\mathds{E}}
\newcommand{\oksendal}{\O{}ksendal}
\newcommand{\lO}{\mathcal{O}}
\newcommand{\1}{\mathds{1}}
\newcommand{\B}{\mathcal{B}}
\newcommand{\A}{\mathcal{A}}
\renewcommand{\bar}[1]{\overline{#1}}
\newcommand{\intd}[1]{\, \mathrm{d} #1}
\renewcommand{\d}{\mathrm{d}}
\newcommand{\var}{\text{Var}}
\newcommand{\Var}{\text{Var}}
\newcommand{\Cov}{\text{Cov}}
\newcommand{\TOL}{\text{TOL}}
\newcommand{\cst}{\text{CST }}
\newcommand{\norm}[1]{\left\lVert#1\right\rVert}
\newcommand{\diff}[2]{\frac{\mathrm{d} #1}{\mathrm{d} #2}}
\newcommand{\difftwo}[2]{\frac{\mathrm{d}^2 #1}{\mathrm{d} #2^2}}
\newcommand{\figref}[1]{Fig. \ref{#1}}
\let\P\relax
\newcommand{\P}{\mathbb{P}}
\newcommand{\eps}{\varepsilon}
\newcommand{\p}{\mathfrak{p}}
\newcommand{\diag}{\text{diag}}
\newcommand{\sign}{\text{sign}}
\newcommand{\tol}{\text{TOL}}

\newtheorem{theorem}{Theorem}
\newtheorem{corollary}{Corollary}[theorem]
\newtheorem{lemma}[theorem]{Lemma}
\graphicspath{ {Figure/} }

\allowdisplaybreaks

\setlength\parskip{0cm}
\setlength\parindent{0cm}

\begin{document}
	
	\begin{flushright}
		\begin{tabular}{|l|c|}
			\hline
			Leon Wilkosz & 193101 \\ \hline
		\end{tabular}
	\end{flushright}
	\begin{center}
		\Large \textbf{AMCS 312}\\ \medskip
		\large\textbf{Homework 1}\\
	\end{center}
	\medskip
	\hrule
	\vspace{10pt}

        \section*{Exercise 1}
		 
		\null\newpage
		
        \section*{Exercise 2}
		\subsection*{(a)}
We assume the setting to be as in the homework sheet and the latencies as in the slides. Let us assume that we desire to wash $n$ loads of laundry. Without parallelization, the time to complete this task is given by
\[
T_1 = n \cdot (30 + 40 + 20) \; \text{minutes} \; = 90n \; \text{minutes}.
\]
The next student always starts drying when the previous student finished drying. Hence, under optimal execution, $n$ students solve the task in
\[
T_n = 30 + n \cdot 40 + 20 \; \text{minutes}.
\]
Thus, the speedup is given as follows and grows asymptotically: 
\[
\frac{T_1}{T_n} =\frac{9n}{3 + 4n + 2} \rightarrow \frac{9}{4} = 2.25 \; \text{ as } n \to \infty.
\]
Independent of $n$, this is executed on three processors (washing, drying, folding). Hence, the parallel efficiency is given by,
\[
E = \frac{T_1}{3 T_n} = \frac{3n}{3 + 4n + 2} \rightarrow 0.75 \; \text{ as } \; n \to \infty.
\]
The maximum parallel efficiency is thus (asymptotically) $0.75$.

\subsection*{(b)}
\FloatBarrier
We obtain the following values for the parallel efficiency:

\begin{center}
\begin{tabular}{c|c|c|c|c|c|c|c|c|c|c|c|c|}
No. of students & 1 & 2 & 3 & 4 & 5 & 6 & 7 & 8 & 9 & 10 & 11 & 12 \\ 
\hline 
Parallel efficiency & $0.333$ & $0.461$ & $0.529$ & $0.571$ & $0.6$ & $0.62$ & $0.636$ & $0.648$ & $0.658$ & $0.666$ & $0.673$ & $0.679$ \\ 
\hline 
\end{tabular}
\end{center}

With twelve students we are in the $10\%$ range, i.e. $\frac{0.75 - 0.679}{0.75} \leq 0.1$.


		\null\newpage
		
        \section*{Exercise 3}
		\subsection*{(a)}
With $c(p) = c$ we find
\[
\partial_p T(p, n) = - \frac{a(n)}{p^{2}} \overset{!}{=} 0, 
\]
i.e. we should ideally use an infinite number of processors.

\subsection*{(b)}
With $c(p) = \log(p)$ we find
\[
\partial_p T(p, n) = - \frac{a(n)}{p^{2}} + \frac{1}{p} \overset{!}{=} 0, 
\]
i.e. $p = a(n)$.

\subsection*{(c)}
With $c(p) = \sqrt{p}$ we find
\[
\partial_p T(p, n) = - \frac{a(n)}{p^{2}} + \frac{1}{2 \sqrt{p}} \overset{!}{=} 0, 
\]
i.e. $p = (2a(n))^{\frac{2}{3}}$.

\subsection*{(d)}
With $c(p) = p$, we find
\[
\partial_p T(p, n) = - \frac{a(n)}{p^{2}} + 1 \overset{!}{=} 0, 
\]
i.e. $p = \sqrt{a(n)}$.

		\null\newpage
		
        \section*{Exercise 4}
        \subsection*{(a)}
Following the lecture slides on the derivation we have the discrete mass conservation
\begin{align*}
\frac{\Delta x \Delta y \Delta z}{\Delta t} (\rho^{l + 1}_{ijk} - \rho^l_{ijk}) = & \; - \Delta y \Delta z\left[ (\rho u)|^l_{i + 1/2, jk} - (\rho u)|^l_{i - 1/2,jk} \right] \\
& - \Delta x \Delta z \left[ (\rho v)|_{i, j + 1/2, k}^l - (\rho v)|_{i, j - 1/2, k} \right] \\
& - \Delta x \Delta y \left[ (\rho w)|_{ij, k + 1/2}^l - (\rho w)|^l_{ij, k - 1/2} \right].
\end{align*}
To estimate the staggered quantities on the right hand side we average over the neighboring cells:
\[
(\rho u)|^l_{i + 1/2, jk} \approx \frac{1}{2} \left[( \rho u)|_{ijk}^l + (\rho u)|^l_{i + 1, jk} \right] \approx \frac{1}{2} \left[ \rho_{ijk}^l u_{ijk}^l + \rho^l_{i + 1, jk} u^l_{i + 1, jk} \right],
\]
and analogous for the other terms. Substituting this into the discrete mass conservation from above, we obtain
\begin{align*}
\frac{\Delta x \Delta y \Delta z}{\Delta t} (\rho^{l + 1}_{ijk} - \rho^l_{ijk}) = & \; - \frac{\Delta y \Delta z}{2}\left[ \rho^l_{i + 1, jk} u^l_{i + 1, jk} -  \rho^l_{i - 1, jk} u^l_{i - 1, jk} \right] \\
& - \frac{\Delta x \Delta z}{2} \left[ \rho^l_{i, j + 1,k} v^l_{i, j + 1, k} -  \rho^l_{i, j - 1, k} v^l_{i, j - 1, k} \right] \\
& - \frac{\Delta x \Delta y}{2} \left[ \rho^l_{ij, k + 1} w^l_{ij, k + 1} -  \rho^l_{ij, k - 1} w^l_{ij, k - 1} \right].
\end{align*}
Assuming that the density is known at time level $l$ we can write down the update step as
\begin{align*}
\rho^{l + 1}_{ijk} = & \; \rho^l_{ijk} - \frac{\Delta t}{2\Delta x}\left[ \rho^l_{i + 1, jk} u^l_{i + 1, jk} -  \rho^l_{i - 1, jk} u^l_{i - 1, jk} \right] \\
& - \frac{\Delta t}{2\Delta y} \left[ \rho^l_{i, j + 1,k} v^l_{i, j + 1, k} -  \rho^l_{i, j - 1, k} v^l_{i, j - 1, k} \right] \\
& - \frac{\Delta t}{2 \Delta z} \left[ \rho^l_{ij, k + 1} w^l_{ij, k + 1} -  \rho^l_{ij, k - 1} w^l_{ij, k - 1} \right].
\end{align*}

\subsection*{(b)} There are three types of special cases at the boundary. The first one is the corner case. Here, the cell to be updated is located in one of the four corners of the cube and three ghost cells need to be introduced to perform the time update, since only three of the six cells required to do the update are part of the domain.

The second type is the edge case. Here, the cell to be updated is located in on of the twelve edges of the cubed domain but not in the corners. We need to introduce two ghost cells and density values at these to complete the update step.

The third type is the side case. Here, the cell to be updated is located at one of the six sides of the cubed domain. We need to introduce one ghost cell and a density value for it to make the update step.

There are different possibilities to set values for these ghost cells. They should come from the specific application as meaningful boundary condition. Without having the authority, since I am not a physicist nor an engineer, I could imagine that periodic boundary conditions could make sense here due to conservation of mass. Or one could set them to the same value as the closest cell inside the domain. Since $\rho$ is after all a continuous function, it makes sense that when $\Delta x$ is small, these values have to be close anyway.

        \null\newpage
        
        \section*{Exercise 5}
        \subsection*{(a)}
Let $A$ be the discrete laplacian as given on slide 22 from unit 2. We find
\begin{align*}
Ac = \left( \begin{matrix} 1 \\ 0 \\ 0 \\ 0 \\ 0 \\ 0 \\ 1 \end{matrix} \right), \; Ar = \left( \begin{matrix} -4 \\ 0 \\ 0 \\ 0 \\ 0 \\ 0 \\ -4 \end{matrix} \right) , \; Ap & = \left( \begin{matrix} 14 \\ -2 \\ -2 \\ -2 \\ -2 \\ -2 \\ 14 \end{matrix} \right), \; As = \left( \begin{matrix} -1 \\ 0 \\ -1 \\ 0 \\ 1 \\ 0 \\ 1 \end{matrix} \right), \; Ao = \left( \begin{matrix} -3 \\ 4 \\ -4 \\ 4 \\ -4 \\ 4 \\ -3 \end{matrix} \right).
\end{align*}
The continuous laplace operator in one dimension is simply the second derivative. Hence, the discretization gives us an approximation of the second derivative, which is zero for the constant, zero for the ramp, constant for the parabola, nonsense for the step because this one is not differentiable at zero (not even continuous ...), and another ocsillatory function for an oscillatory function (second derivative of $\sin$ gives $\sin$ again). The scaling and the signs are a bit off above for the second derivatives. This is because to be truly a discretization of the continuous laplace operator, one needs a factor of $-\frac{1}{4}$ in front of the matrix $A$ on slide $22$.

\subsection*{(b)}
With $A$ the discrete Laplacian we have $A^{-1} l_4 = u_4$. This is because by definition of the inverse $A^{-1} A = I$, where $I$ denotes the identity matrix. Moreover,
\[
A^{-1} u_4 = A^{-1}_{:, 4} = \frac{1}{8} \left( \begin{matrix} 4 \\ 8 \\ 12 \\ 16 \\ 12 \\ 8 \\ 4 \end{matrix} \right).
\]
This is because $A^{-1} = A^{-1} I$. One could argue, that integrating a Dirac distribution gives a step function and integrating a step function gives a ramp.

\subsection*{(c)}
Applying the inverse of the laplacian to the constant gives
\[
A^{-1} c = \left( \begin{matrix} 3.5 \\ 6 \\ 7.5 \\ 8 \\ 7.5 \\ 6 \\ 3.5 \end{matrix} \right),
\]
which is a parabola. This makes sense because its second derivative is constant.

        \null\newpage
        
        \section*{Exercise 6}
        %\subsection*{(a)}
The Top500 HPL List is ranked by the High-Performance Linpack (HPL) Benchmark. The latter uses the LU decomposition algorithm with partial pivoting to solve a dense system of linear equations. The associated matrix has randomly distributed elements between $-1$ and $1$. The performance is measured in flop/s and the highest sustained number of flops/s achievable during this computation determines a supercomputer's ranking in the list.

The Top124 list is ranked by the High Performance Conjugate Gradient (HPCG) benchmark. the latter uses the conjugate gradient algorithm to solve large, sparse, symmetric, positive-definite linear systems. Performance is measured in flop/s and the highest sustained number of flops achievable during this computation determines a supercomputer's ranking in the list.

\subsection*{(b)}

\begin{tabular}{|c|c|c|c|c|c|c|}
Sum & HPL list & Green list & Manufacturer & Model & Nickname & Country \\ 
\hline 
19 & 1 & 18 & HPE & HPE Cray EX255a & El Capitan & USA \\ 
\hline 
21 & 7 & 14 & HPE & HPE Cray EX254n & Alps & Switzerland \\ 
\hline 
22 & 10 & 12 & HPE & HPE Cray EX255a & Tuolumne & USA \\ 
\hline 
24 & 18 & 6 & ParTec/EVIDEN & BallSequana XH3000 & JETI & Germany \\ 
\hline 
24 & 2 & 22 & HPE & HPE Cray EX235a & Frontier & USA \\ 
\hline 
26 & 5 & 21 & HPE & HPE Cray EX235a & HPC6 & Italy \\ 
\hline 
29 & 13 & 16 & HPE & HPE Cray Ex254n & Venado & USA \\ 
\hline 
33 & 20 & 13 & HPE & HPE Cray Ex255a & El Dorado & USA \\ 
\hline 
33 & 8 & 25 & HPE & HPE Cray Ex235a & LUMI & Finland \\ 
\hline 
49 & 30 & 19 & HPE & HPE Cray EX235a & Adastra & France \\ 
\hline 
\end{tabular} 



	    \null\newpage
        
		\section*{Exercise 7}
        %\subsection*{(a)}
Nine challenges:
\begin{enumerate}
\item Scalable multicore systems bring a growing cost of communication relative to computation. Especially across different nodes (multicore processors) the cost of data transfer becomes large
\item Static distribution of tasks and adaptive multiscale algorithms introduce load imbalances from dynamically changing computation
\item Since 32-bit (single precision) operations are at least twice as fast as 64-bit operations on modern architectures and have smaller storage and memory traffic, we need mixed precision algorithms to utilize heterogenous hardware effectively.
\item Memory movement is increasingly expensive compared with the cost of computation, need to develop and study communication avoiding algorithms
\item Numerical libraries need to be able to adapt to possibly heterogeneous environments to remain the same for the user but be consistent independent of scale and processor heterogeneity. (Auto-tuning)
\item Due to scale and complexity of supercomputer architectures, faults become often and current restarting techniques are not scalable to highly parallel systems. New faults will occur  before the application can be restarted. (Fault Tolerance and Robustness)
\item Energy consumption is becoming a problem. Depends on hardware and software.
\item One needs to study sensitivity of high fidelity models to parameter variability and uncertainty.
\item With more powerful machines the dimensions of the problems increase, but algorithms that work well for moderate dimensions might fail in subtle ways for larger problems. Reproducability independent of scale and number of cores is an open issue.
\end{enumerate}

\subsection*{(b)}
In today's exascale environment challenge 6 becomes much more critical. Most restarting techniques are not scalable to highly parallel systems, so that a specific application might not be able to restart before the next fault occurs. This results in a program getting stuck. For algorithms in computing molecular dynamics this is a hard challenge because these computations are typically long-running and data intensive. An adaption could be to use redundant computing. If some redundant computations get lost due to a fault they don't need to be redone and the algorithm can restart quicker. Otherwise, one can let algorithms adaptively run on a dynamic number of cores, so that in case of a fault the algorithm can continue running after a unit of the computer is shut off. This would include ongoing error measurements and decisions on whether possibly corrupted data should be excluded combined with modular composition of the simulation.
        \null\newpage
        
        \section*{Exercise 8}
        %\subsection*{(a)}
The Mandates are as follows:
\begin{enumerate}
\item Better resolve a model's full, natural range of length or time scales
\item Accommodate physical effects with greater fidelity
\item Allow the model degrees of freedom in all relevant dimensions
\item Better isolate artificial boundary conditions
\item Combine multiple complex models
\item Solve an inverse problem or perform data assimilation
\item Perform otimization or control
\item Quantify uncertainty
\item Accomplish predictions without physical models using statistical models based on large data sets
\end{enumerate}

\subsection*{(b)}
Accomodating physical effects with greater fidelity is an important opportunity in aerodynamic modeling for e.g. aircraft design. This includes the modeling of turbulent flows and shock waves. Still, many costly experiments need to be carried out in the wind tunnel including people and material as their virtual numerical pendants in computational fluid dynamics are not feasible. This includes especially numerical simulations of high-Reynolds-number flows. Also, wind tunnel experiments are physically not optimal to see all possible effects on a moving airplane as even here many approximations need to be made to translate the results into a real life situation. One of the potentials of HPC in this area is that given enough computation power, simulations might become more accurate than wind tunnel experiments. This raises a second important point as it will not only safe costs in the production, but it will make airplanes safer and more efficient in practice.
        \null\newpage
        
\end{document}














