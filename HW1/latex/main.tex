\documentclass[
a4paper,
]{article}
\usepackage{lmodern}
\usepackage[utf8]{inputenc}
\usepackage[T1]{fontenc}


\addtolength\jot{1ex}


\usepackage[utf8]{inputenc}
\usepackage{amsmath}
\usepackage{amsthm}
\usepackage{listings}
\usepackage{color}
%\usepackage{german}
%\usepackage{capt-of}
\usepackage[font=small,
figurewithin=section,labelsep=space]{caption}
\usepackage{amssymb}
\usepackage{tabto}
\usepackage{graphicx}
\usepackage{enumerate}
\usepackage{cite}
\usepackage{tikz-cd}
\usepackage{faktor}
\usepackage{verbatim}
\usepackage{dsfont}
\usepackage{svg}
\usepackage{longtable}
\usepackage{bbm}
\usepackage{mathtools}
%\usepackage[vcentermath]{youngtab}
\usepackage[left=3cm, top=2cm, bottom=2.5cm, right=3cm]{geometry}
\usepackage{appendix}
\usepackage[english]{hyperref}
\usepackage{cleveref}
\usepackage{placeins}
\usepackage{bm}
\usepackage{array, diagbox}
\setcounter{MaxMatrixCols}{20}

\let\div\relax
\DeclareMathOperator{\div}{div}
\newcommand{\dd}[1]{\frac{\partial}{\partial #1}}
\newcommand{\ddi}[2]{\frac{\partial #1}{\partial #2}}
\newcommand{\ddtwo}[1]{\frac{\partial^2}{\partial #1^2}}
\newcommand{\ddtwoi}[2]{\frac{\partial^2 #1}{\partial #2^2}}
\newcommand{\ddtwod}[3]{\frac{\partial^2 #1}{\partial #2 \partial #3}}
\DeclareMathOperator{\tr}{Tr}
\newcommand{\kl}{\left(}
\newcommand{\kr}{\right)}
\newcommand{\el}{\left[}
\newcommand{\er}{\right]}
\newcommand{\R}{\mathbb{R}}
\newcommand{\C}{\mathbb{C}}
\newcommand{\Z}{\mathbb{Z}}
\newcommand{\N}{\mathbb{N}}
\newcommand{\cl}{\mathcal{L}}
\newcommand{\cf}{\mathcal{F}}
\newcommand{\cb}{\mathcal{B}}
\newcommand{\Ocal}{\mathcal{O}}
\newcommand{\simd}{\overset{\text{d}}{\sim}}
\newcommand{\conp}{\xrightarrow{P}}
\newcommand{\cond}{\xrightarrow{\text{d}}}
\newcommand{\conas}{\xrightarrow{\text{f.s.}}}
\newcommand{\conlp}{\xrightarrow{L^p}}
\newcommand{\conlpx}[1]{\xrightarrow{L^{#1}}}
\newcommand{\conv}{\xrightarrow{\text{v}}}
\newcommand{\E}{\mathds{E}}
\newcommand{\oksendal}{\O{}ksendal}
\newcommand{\lO}{\mathcal{O}}
\newcommand{\1}{\mathds{1}}
\newcommand{\B}{\mathcal{B}}
\newcommand{\A}{\mathcal{A}}
\renewcommand{\bar}[1]{\overline{#1}}
\newcommand{\intd}[1]{\, \mathrm{d} #1}
\renewcommand{\d}{\mathrm{d}}
\newcommand{\var}{\text{Var}}
\newcommand{\Var}{\text{Var}}
\newcommand{\Cov}{\text{Cov}}
\newcommand{\TOL}{\text{TOL}}
\newcommand{\cst}{\text{CST }}
\newcommand{\norm}[1]{\left\lVert#1\right\rVert}
\newcommand{\diff}[2]{\frac{\mathrm{d} #1}{\mathrm{d} #2}}
\newcommand{\difftwo}[2]{\frac{\mathrm{d}^2 #1}{\mathrm{d} #2^2}}
\newcommand{\figref}[1]{Fig. \ref{#1}}
\let\P\relax
\newcommand{\P}{\mathbb{P}}
\newcommand{\eps}{\varepsilon}
\newcommand{\p}{\mathfrak{p}}
\newcommand{\diag}{\text{diag}}
\newcommand{\sign}{\text{sign}}
\newcommand{\tol}{\text{TOL}}

\newtheorem{theorem}{Theorem}
\newtheorem{corollary}{Corollary}[theorem]
\newtheorem{lemma}[theorem]{Lemma}
\graphicspath{ {Figure/} }

\allowdisplaybreaks

\setlength\parskip{0cm}
\setlength\parindent{0cm}

\begin{document}
	
	
	\begin{flushright}
		\begin{tabular}{|l|c|}
			\hline
			Leon Wilkosz & 193101 \\ \hline
		\end{tabular}
	\end{flushright}
	\begin{center}
		\Large \textbf{AMCS 312}\\ \medskip
		\large\textbf{Homework 1}\\
	\end{center}
	\medskip
	\hrule
	\vspace{10pt}

        \section*{Exercise 1}
		\subsection*{(a)}
\FloatBarrier
We compute the speedup $\frac{T_1}{T_p}$ for different combinations of non-parallelizable fractions of code and the number of processors in \Cref{tab:ex1}. The graph of the cross section along the middle row is found in \Cref{FixProc} and the graph of the cross section along the middle column is found in \Cref{FixFrac}. We choose to evaluate the Speedup in the number of processors in an exponential scale, since for a large number of processors, relatively small changes become less important. On the other hand, we choose equally spaced $a$, since the whole range seems interesting.
\\

The results of this exercise give us a realistic expectation of what to expect from the speedups possible with parallelization. We see in \Cref{FixProc} that it highly depends on the fraction of parallelizable code and that already a little fraction of non-parallelizable code is enough to strongly dampen the speedup. Also, in \Cref{FixFrac} we see that given a fixed fraction of non-parallelizable code, it does not make sense to increase the number of processors at some point as the speedup converges asyptotically to a maximum given through Amdahl's law. The overall conclusion is that it is crucial to develop algorithms with an as large as possible parallelizable fraction to see the benefits of parallel computing and that the latter is not a magic tool that makes every computation efficient.


\begin{table}[h]
\hspace{-0.1cm}
\begin{tabular}{|c|c|c|c|c|c|c|c|c|c|c|c|}
\hline
\diagbox[innerwidth = 2cm, height = 3ex]{p}{a} & $0$ & $0.1$ & $0.2$ & $0.3$ & $0.4$ & $0.5$ & $0.6$ & $0.7$ & $0.8$ & $0.9$ & $1$\\ 
\hline 
$1$ & $1$ & $1$ & $1$ & $1$ & $1$ & $1$ & $1$ & $1$ & $1$ & $1$ & $1$ \\ 
\hline 
$2$ & $2$ & $1.818$ & $1.666$ & $1.538$ & $1.428$ & $1.333$ & $1.25$ & $1.176$ & $1.111$ & $1.052$ & $1$ \\ 
\hline 
$4$ & $4$ & $3.076$ & $2.5$ & $2.105$ & $1.818$ & $1.6$ & $1.428$ & $1.29$ & $1.176$ & $1.081$ & $1$ \\ 
\hline 
$8$ & $8$ & $4.705$ & $3.333$ & $2.58$ & $2.105$ & $1.777$ & $1.538$ & $1.355$ & $1.212$ & $1.095$ & $1$ \\ 
\hline 
$16$ & $16$ & $6.4$ & $4$ & $2.909$ & $2.285$ & $1.882$ & $1.6$ & $1.391$ & $1.23$ & $1.103$ & $1$ \\ 
\hline 
$32$ & $32$ & $7.804$ & $4.444$ & $3.106$ & $2.388$ & $1.939$ & $1.632$ & $1.409$ & $1.24$ & $1.107$ & $1$ \\ 
\hline 
$64$ & $64$ & $8.767$ & $4.705$ & $3.216$ & $2.442$ & $1.969$ & $1.649$ & $1.419$ & $1.245$ & $1.109$ & $1$ \\ 
\hline 
$128$ & $128$ & $9.343$ & $4.848$ & $3.273$ & $2.471$ & $1.984$ & $1.658$ & $1.423$ & $1.247$ & $1.11$ & $1$ \\ 
\hline 
$256$ & $256$ & $9.66$ & $4.923$ & $3.303$ & $2.485$ & $1.992$ & $1.662$ & $1.426$ & $1.248$ & $1.110$ & $1$ \\ 
\hline 
$512$ & $512$ & $9.827$ & $4.961$ & $3.318$ & $2.492$ & $1.996$ & $1.664$ & $1.427$ & $1.249$ & $1.11$ & $1$ \\ 
\hline 
$1024$ & $1024$ & $9.912$ & $4.98$ & $3.325$ & $2.496$ & $1.998$ & $1.665$ & $1.427$ & $1.249$ & $1.11$ & $1$ \\ 
\hline 
\end{tabular}
\caption{Speedups for different combinations of non-parallelizable fractions and processor numbers}
\label{tab:ex1}
\end{table}


\begin{figure}[h]
\centering
\includegraphics[width=0.9\textwidth]{FixProc.pdf}
\caption{Speedup for a fixed number of processors over the fraction of non-parallelizable code}
\label{FixProc}
\end{figure}

\begin{figure}[h]
\centering
\includegraphics[width=0.9\textwidth]{FixFrac.pdf}
\caption{Speedup for a fixed fraction of non-parallelizable code over the number of processors}
\label{FixFrac}
\end{figure}



\begin{comment}
[0.  0.1 0.2 0.3 0.4 0.5 0.6 0.7 0.8 0.9 1. ]
[   1    2    4    8   16   32   64  128  256  512 1024]
[[1.00000000e+00 1.00000000e+00 1.00000000e+00 1.00000000e+00
  1.00000000e+00 1.00000000e+00 1.00000000e+00 1.00000000e+00
  1.00000000e+00 1.00000000e+00 1.00000000e+00]
 [2.00000000e+00 1.81818182e+00 1.66666667e+00 1.53846154e+00
  1.42857143e+00 1.33333333e+00 1.25000000e+00 1.17647059e+00
  1.11111111e+00 1.05263158e+00 1.00000000e+00]
 [4.00000000e+00 3.07692308e+00 2.50000000e+00 2.10526316e+00
  1.81818182e+00 1.60000000e+00 1.42857143e+00 1.29032258e+00
  1.17647059e+00 1.08108108e+00 1.00000000e+00]
 [8.00000000e+00 4.70588235e+00 3.33333333e+00 2.58064516e+00
  2.10526316e+00 1.77777778e+00 1.53846154e+00 1.35593220e+00
  1.21212121e+00 1.09589041e+00 1.00000000e+00]
 [1.60000000e+01 6.40000000e+00 4.00000000e+00 2.90909091e+00
  2.28571429e+00 1.88235294e+00 1.60000000e+00 1.39130435e+00
  1.23076923e+00 1.10344828e+00 1.00000000e+00]
 [3.20000000e+01 7.80487805e+00 4.44444444e+00 3.10679612e+00
  2.38805970e+00 1.93939394e+00 1.63265306e+00 1.40969163e+00
  1.24031008e+00 1.10726644e+00 1.00000000e+00]
 [6.40000000e+01 8.76712329e+00 4.70588235e+00 3.21608040e+00
  2.44274809e+00 1.96923077e+00 1.64948454e+00 1.41906874e+00
  1.24513619e+00 1.10918544e+00 1.00000000e+00]
 [1.28000000e+02 9.34306569e+00 4.84848485e+00 3.27365729e+00
  2.47104247e+00 1.98449612e+00 1.65803109e+00 1.42380423e+00
  1.24756335e+00 1.11014744e+00 1.00000000e+00]
 [2.56000000e+02 9.66037736e+00 4.92307692e+00 3.30322581e+00
  2.48543689e+00 1.99221790e+00 1.66233766e+00 1.42618384e+00
  1.24878049e+00 1.11062907e+00 1.00000000e+00]
 [5.12000000e+02 9.82725528e+00 4.96124031e+00 3.31821128e+00
  2.49269718e+00 1.99610136e+00 1.66449935e+00 1.42737664e+00
  1.24938995e+00 1.11087004e+00 1.00000000e+00]
  
  
 [1.02400000e+03 9.91287512e+00 4.98054475e+00 3.32575512e+00
  2.49634325e+00 1.99804878e+00 1.66558230e+00 1.42797378e+00
  1.24969490e+00 1.11099056e+00 1.00000000e+00]]
  \end{comment}
  
  \FloatBarrier
  
  
  
\subsection*{(b)}
\FloatBarrier
When $a$ is the fraction of non-parallelizable code, Ahmdals law describes the time $T_p$ to run on $p$ processors as
\[
T_P = (1 - a)\frac{T_1}{p} + a T_1,
\]
where $T_1$ is the time needed to solve the problem on $1$ node and $p$ is the number of nodes. A computation derives the parallel efficiency as
\begin{align*}
	E & = \frac{T_1}{pT_p} = \frac{T_1}{p(1 - a)\frac{T_1}{p} + ap T_1} = \frac{1}{1 - a + ap}.
\end{align*}
\FloatBarrier

\subsection*{(c)}
\FloatBarrier
As an example we can consider the method of lines for the solution $u(x, t)$ of a scalar hyperbolic PDE with two dimensional space component $x \in \Omega \subset \R^2$ and one dimensional time component $t$. Due to finite speed of information propagation, coming from hyperbolicity, it is enough to use $\Delta t = \Ocal(\Delta x)$ to ensure stability. We can and will therefore assume for now that $\Delta x = \Ocal(\frac{1}{n})$ and $\Delta t = \Ocal\left(\frac{1}{n}\right)$. Put simply, the method of lines discretizes space and computes the time update through independent ODEs. The cost for computing these updates with Euler's method is $\Ocal\left(\frac{1}{\Delta t}\right) = \Ocal(n)$. Further, this time update is a sequential operation that is not parallelizable. Since space is two dimensional here, this has to be done independently $\Ocal\left(\frac{1}{(\Delta x)^2}\right) = \Ocal(n^2)$ times, once at each grid point. This can be done completely in parallel. Concluding, the overall cost of the problem grows as
\[
\Ocal(n^2) \cdot \Ocal(n) = \Ocal(n^3),
\]
whereas the non-parallelizable part of it grows as
\[
\Ocal(n).
\]
Translating this into the language of the previous parts, the non-paralellizable fraction $a$ evolves as
\[
a = \Ocal\left(\frac{n}{n^3}\right) = \Ocal\left(\frac{1}{n^2}\right),
\]
and the parallelizable fraction evolves as
\[
1 - a = \Ocal\left(1 - \frac{1}{n^2}\right).
\]
Choosing an appropriate $p = \Ocal(n^2)$ gives here constant parallel efficiency.
\FloatBarrier

\subsection*{(d)}
\FloatBarrier
Let all the notions be given as in the exercise sheet. We assume that the time necessary to solve the problem without parallelization is $T_1$. The time $T_P$ to run on the maximal number of nodes is then given by summing the parallelized runtimes for the different classes of work:
\[
T_P = \sum_{p = 1}^P a_p \frac{T_1}{p},
\]
The respective speedup is given by
\[
\frac{T_1}{T_P} = \frac{1}{\sum_{p = 1}^P \frac{a_p}{p}}.
\]
\FloatBarrier
  
  
  
  
  
  
  
  
  
  
  
  
  
  
  
  
  
  
  
  
  
  
  
  
  
  
  
  
  
  
  
  
  
  
  
  
  
  
  
  
  
  
  
  
  
  
  
  
  
  
  
		\null\newpage
		
        \section*{Exercise 2}
		\subsection*{(a)}
We assume the setting to be as in the homework sheet and the latencies as in the slides. Let us assume that we desire to wash $n$ loads of laundry. Without parallelization, the time to complete this task is given by
\[
T_1 = n \cdot (30 + 40 + 20) \; \text{minutes} \; = 90n \; \text{minutes}.
\]
The next student always starts drying when the previous student finished drying. Hence, under optimal execution, $n$ students solve the task in
\[
T_n = 30 + n \cdot 40 + 20 \; \text{minutes}.
\]
Thus, the speedup is given as follows and grows asymptotically: 
\[
\frac{T_1}{T_n} =\frac{9n}{3 + 4n + 2} \rightarrow \frac{9}{4} = 2.25 \; \text{ as } n \to \infty.
\]
Independent of $n$, this is executed on three processors (washing, drying, folding). Hence, the parallel efficiency is given by,
\[
E = \frac{T_1}{3 T_n} = \frac{3n}{3 + 4n + 2} \rightarrow 0.75 \; \text{ as } \; n \to \infty.
\]
The maximum parallel efficiency is thus (asymptotically) $0.75$.

\subsection*{(b)}
\FloatBarrier
We obtain the following values for the parallel efficiency:

\begin{center}
\begin{tabular}{c|c|c|c|c|c|c|c|c|c|c|c|c|}
No. of students & 1 & 2 & 3 & 4 & 5 & 6 & 7 & 8 & 9 & 10 & 11 & 12 \\ 
\hline 
Parallel efficiency & $0.333$ & $0.461$ & $0.529$ & $0.571$ & $0.6$ & $0.62$ & $0.636$ & $0.648$ & $0.658$ & $0.666$ & $0.673$ & $0.679$ \\ 
\hline 
\end{tabular}
\end{center}

With twelve students we are in the $10\%$ range, i.e. $\frac{0.75 - 0.679}{0.75} \leq 0.1$.


		\null\newpage
		
        \section*{Exercise 3}
		\subsection*{(a)}
With $c(p) = c$ we find
\[
\partial_p T(p, n) = - \frac{a(n)}{p^{2}} \overset{!}{=} 0, 
\]
i.e. we should ideally use an infinite number of processors.

\subsection*{(b)}
With $c(p) = \log(p)$ we find
\[
\partial_p T(p, n) = - \frac{a(n)}{p^{2}} + \frac{1}{p} \overset{!}{=} 0, 
\]
i.e. $p = a(n)$.

\subsection*{(c)}
With $c(p) = \sqrt{p}$ we find
\[
\partial_p T(p, n) = - \frac{a(n)}{p^{2}} + \frac{1}{2 \sqrt{p}} \overset{!}{=} 0, 
\]
i.e. $p = (2a(n))^{\frac{2}{3}}$.

\subsection*{(d)}
With $c(p) = p$, we find
\[
\partial_p T(p, n) = - \frac{a(n)}{p^{2}} + 1 \overset{!}{=} 0, 
\]
i.e. $p = \sqrt{a(n)}$.

		\null\newpage
		
        \section*{Exercise 4}
         \subsection*{(a)}
 \FloatBarrier
 We compute the Laplacians:
\begin{align*}
& \textbf{1D:} \quad \hat{\varphi}(x) = x - x^2, \quad \Delta \hat{\varphi}(x) = -2\\
& \textbf{2D:} \quad \hat{\varphi}(x, y) = (x - x^2)(y^2 - y), \quad
\Delta \hat{\varphi}(x, y) = -2(y^2 - y) + 2(x - x^2) \\
& \textbf{3D:} \quad \hat{\varphi}(x, y, z) = (x - x^2)(y^2 - y)(z - z^2)\\
& \Delta \hat{\varphi}(x, y, z) = -2(y^2 - y)(z - z^2) + 2(x - x^2)(z - z^2) - 2(x - x^2)(y^2 - y)
\end{align*}
and need to set in the code $f = -\Delta \hat\varphi$ in each dimension as the boundary condition. By direct computation, we can show that with a uniform step size the discrete laplacian at $x \in [0, 1]$ is identical to the true continuous Laplacian. For instance, in 1D we find
\begin{align*}
\frac{\hat\varphi(x + h) - 2\hat\varphi(x) + \hat\varphi(x - h)}{h^2} & = \frac{x + h - (x + h)^2 -2(x - x^2) + x - h - (x - h)^2}{h^2} \\
& = -2 \\
& = \Delta \hat\varphi(x).
\end{align*}
Similar computations can be carried out for 2D and 3D. This means that the discretized equation is already an exact equation for the solution at the grid points. As a consequence, there is no truncation error and what we observe as a numerical error is the KSP residual after the last iteration. Hence, we get very good accuracy very fast as can be seen in \Cref{1D_err}, \Cref{2D_err}, and \Cref{3D_err}.

\begin{table}[h!]
\centering
\begin{tabular}{|c|c|}
\hline
\textbf{Grid Size} & \textbf{Infinity Norm Error} \\
\hline
17 & $5.55112 \times 10^{-17}$ \\
\hline
33 & $1.94289 \times 10^{-16}$ \\
\hline
65 & $5.55112 \times 10^{-16}$ \\
\hline
\end{tabular}
\caption{Numerical Error in Infinity Norm for 1D}
\label{1D_err}
\end{table}

\begin{table}[h!]
\centering
\begin{tabular}{|c|c|}
\hline
\textbf{Grid Size} & \textbf{Infinity Norm Error} \\
\hline
17 $\times$ 17 & $5.23059 \times 10^{-8}$ \\
\hline
33 $\times$ 33 & $2.39077 \times 10^{-7}$ \\
\hline
65 $\times$ 65 & $9.99247 \times 10^{-8}$ \\
\hline
\end{tabular}
\caption{Numerical Error in Infinity Norm for 2D}
\label{2D_err}
\end{table}

\begin{table}[h!]
\centering
\begin{tabular}{|c|c|}
\hline
\textbf{Grid Size} & \textbf{Infinity Norm Error} \\
\hline
17 $\times$ 17 $\times$ 17 & $6.44696 \times 10^{-8}$ \\
\hline
33 $\times$ 33 $\times$ 33 & $4.60936 \times 10^{-8}$ \\
\hline
65 $\times$ 65 $\times$ 65 & $2.84432 \times 10^{-8}$ \\
\hline
\end{tabular}
\caption{Numerical Error in Infinity Norm for 3D}
\label{3D_err}
\end{table}
\FloatBarrier

\subsection*{(b)}
\FloatBarrier
We compute first the Laplacians:
\begin{align*}
&\textbf{1D:} \quad \hat{\varphi}(x) = 3x + \sin(20x), \quad \Delta \hat{\varphi}(x) = -400\sin(20x) \\
&\textbf{2D:} \quad \hat{\varphi}(x, y) = 3x + \sin(20xy), \quad \Delta \hat{\varphi}(x, y) = -400(x^2 + y^2)\sin(20xy) \\
&\textbf{3D:} \quad \hat{\varphi}(x, y, z) = 3x + 3z + \sin(20xyz), \quad \Delta \hat{\varphi}(x, y, z) = -400(x^2y^2 + x^2z^2 + y^2z^2)\sin(20xyz)
\end{align*}
We adjust the code adding the boundary condition $g=\varphi$ and the new right-hand side $f = -\Delta \varphi$. \Cref{err_1D_b}, \Cref{err_2D_b}, and \Cref{err_3D_b} show the numerical error for different dimensions and grid sizes. \Cref{rate_1D_b}, and \Cref{rate_2D_b} show the convergence rates in the sense that the error of the respective grid is divided by the error at the next coarser grid. Since there is no coarser grid, the first row is empty. We expect that the numerical error shrinks by a factor of four every time we double the resolution. Since the quotients in the tables are about 4, we can confirm that we see the convergence rates.


\begin{table}[h!]
\centering
\begin{tabular}{|c|c|}
\hline
Grid Size & Infinity Norm Error \\
\hline
17 & 0.250283 \\
33 & 0.058908 \\
65 & 0.01456 \\
\hline
\end{tabular}
\caption{1D Numerical Errors in Infinity Norm}
\label{err_1D_b}
\end{table}

\begin{table}[h!]
\centering
\begin{tabular}{|c|c|}
\hline
Grid Size & Infinity Norm Error \\
\hline
17x17 & 0.118495 \\
33x33 & 0.0279685 \\
65x65 & 0.00650328 \\
\hline
\end{tabular}
\caption{2D Numerical Errors in Infinity Norm}
\label{err_2D_b}
\end{table}

\begin{table}[h!]
\centering
\begin{tabular}{|c|c|}
\hline
Grid Size & Infinity Norm Error \\
\hline
17x17x17 & 0.0794387 \\
33x33x33 & 0.0207824 \\
65x65x65 & 0.00511722 \\
\hline
\end{tabular}
\caption{3D Numerical Errors in Infinity Norm}
\label{err_3D_b}
\end{table}

\begin{table}[h!]
\centering
\begin{tabular}{|c|c|}
\hline
Grid Size & Convergence Rate \\
\hline
17 &  \\
33 & 4.247 \\
65 & 4.044 \\
\hline
\end{tabular}
\caption{1D Convergence Rates}
\label{rate_1D_b}
\end{table}

\begin{table}[h!]
\centering
\begin{tabular}{|c|c|}
\hline
Grid Size & Convergence Rate \\
\hline
17x17 &  \\
33x33 & 4.236 \\
65x65 & 4.299 \\
\hline
\end{tabular}
\caption{2D Convergence Rates}
\label{rate_2D_b}
\end{table}

\begin{table}[h!]
\centering
\begin{tabular}{|c|c|}
\hline
Grid Size & Convergence Rate \\
\hline
17x17x17 &  \\
33x33x33 & 3.819 \\
65x65x65 & 4.062 \\
\hline
\end{tabular}
\caption{3D Convergence Rates}
\label{rate_2D_b}
\end{table}







\FloatBarrier



        \null\newpage
        
        \section*{Exercise 5}
        \subsection*{(a)}
Let $A$ be the discrete laplacian as given on slide 22 from unit 2. We find
\begin{align*}
Ac = \left( \begin{matrix} 1 \\ 0 \\ 0 \\ 0 \\ 0 \\ 0 \\ 1 \end{matrix} \right), \; Ar = \left( \begin{matrix} -4 \\ 0 \\ 0 \\ 0 \\ 0 \\ 0 \\ -4 \end{matrix} \right) , \; Ap & = \left( \begin{matrix} 14 \\ -2 \\ -2 \\ -2 \\ -2 \\ -2 \\ 14 \end{matrix} \right), \; As = \left( \begin{matrix} -1 \\ 0 \\ -1 \\ 0 \\ 1 \\ 0 \\ 1 \end{matrix} \right), \; Ao = \left( \begin{matrix} -3 \\ 4 \\ -4 \\ 4 \\ -4 \\ 4 \\ -3 \end{matrix} \right).
\end{align*}
The continuous laplace operator in one dimension is simply the second derivative. Hence, the discretization gives us an approximation of the second derivative, which is zero for the constant, zero for the ramp, constant for the parabola, nonsense for the step because this one is not differentiable at zero (not even continuous ...), and another ocsillatory function for an oscillatory function (second derivative of $\sin$ gives $\sin$ again). The scaling and the signs are a bit off above for the second derivatives. This is because to be truly a discretization of the continuous laplace operator, one needs a factor of $-\frac{1}{4}$ in front of the matrix $A$ on slide $22$.

\subsection*{(b)}
With $A$ the discrete Laplacian we have $A^{-1} l_4 = u_4$. This is because by definition of the inverse $A^{-1} A = I$, where $I$ denotes the identity matrix. Moreover,
\[
A^{-1} u_4 = A^{-1}_{:, 4} = \frac{1}{8} \left( \begin{matrix} 4 \\ 8 \\ 12 \\ 16 \\ 12 \\ 8 \\ 4 \end{matrix} \right).
\]
This is because $A^{-1} = A^{-1} I$. One could argue, that integrating a Dirac distribution gives a step function and integrating a step function gives a ramp.

\subsection*{(c)}
Applying the inverse of the laplacian to the constant gives
\[
A^{-1} c = \left( \begin{matrix} 3.5 \\ 6 \\ 7.5 \\ 8 \\ 7.5 \\ 6 \\ 3.5 \end{matrix} \right),
\]
which is a parabola. This makes sense because its second derivative is constant.

        \null\newpage
        
        \section*{Exercise 6}
        \subsection*{(a)}
The Top500 HPL List is ranked by the High-Performance Linpack (HPL) Benchmark. The latter uses the LU decomposition algorithm with partial pivoting to solve a dense system of linear equations. The associated matrix has randomly distributed elements between $-1$ and $1$. The performance is measured in flop/s and the highest sustained number of flops/s achievable during this computation determines a supercomputer's ranking in the list.

The Top124 list is ranked by the High Performance Conjugate Gradient (HPCG) benchmark. the latter uses the conjugate gradient algorithm to solve large, sparse, symmetric, positive-definite linear systems. Performance is measured in flop/s and the highest sustained number of flops achievable during this computation determines a supercomputer's ranking in the list.

\subsection*{(b)}

\begin{tabular}{|c|c|c|c|c|c|c|}
Sum & HPL list & Green list & Manufacturer & Model & Nickname & Country \\ 
\hline 
19 & 1 & 18 & HPE & HPE Cray EX255a & El Capitan & USA \\ 
\hline 
21 & 7 & 14 & HPE & HPE Cray EX254n & Alps & Switzerland \\ 
\hline 
22 & 10 & 12 & HPE & HPE Cray EX255a & Tuolumne & USA \\ 
\hline 
24 & 18 & 6 & ParTec/EVIDEN & BallSequana XH3000 & JETI & Germany \\ 
\hline 
24 & 2 & 22 & HPE & HPE Cray EX235a & Frontier & USA \\ 
\hline 
26 & 5 & 21 & HPE & HPE Cray EX235a & HPC6 & Italy \\ 
\hline 
29 & 13 & 16 & HPE & HPE Cray Ex254n & Venado & USA \\ 
\hline 
33 & 20 & 13 & HPE & HPE Cray Ex255a & El Dorado & USA \\ 
\hline 
33 & 8 & 25 & HPE & HPE Cray Ex235a & LUMI & Finland \\ 
\hline 
49 & 30 & 19 & HPE & HPE Cray EX235a & Adastra & France \\ 
\hline 
\end{tabular} 



	    \null\newpage
        
		\section*{Exercise 7}
        \subsection*{(a)}
Nine challenges:
\begin{enumerate}
\item Scalable multicore systems bring a growing cost of communication relative to computation. Especially across different nodes (multicore processors) the cost of data transfer becomes large
\item Static distribution of tasks and adaptive multiscale algorithms introduce load imbalances from dynamically changing computation
\item Since 32-bit (single precision) operations are at least twice as fast as 64-bit operations on modern architectures and have smaller storage and memory traffic, we need mixed precision algorithms to utilize heterogenous hardware effectively.
\item Memory movement is increasingly expensive compared with the cost of computation, need to develop and study communication avoiding algorithms
\item Numerical libraries need to be able to adapt to possibly heterogeneous environments to remain the same for the user but be consistent independent of scale and processor heterogeneity. (Auto-tuning)
\item Due to scale and complexity of supercomputer architectures, faults become often and current restarting techniques are not scalable to highly parallel systems. New faults will occur  before the application can be restarted. (Fault Tolerance and Robustness)
\item Energy consumption is becoming a problem. Depends on hardware and software.
\item One needs to study sensitivity of high fidelity models to parameter variability and uncertainty.
\item With more powerful machines the dimensions of the problems increase, but algorithms that work well for moderate dimensions might fail in subtle ways for larger problems. Reproducability independent of scale and number of cores is an open issue.
\end{enumerate}

\subsection*{(b)}
In today's exascale environment challenge 6 becomes much more critical. Most restarting techniques are not scalable to highly parallel systems, so that a specific application might not be able to restart before the next fault occurs. This results in a program getting stuck. For algorithms in computing molecular dynamics this is a hard challenge because these computations are typically long-running and data intensive. An adaption could be to use redundant computing. If some redundant computations get lost due to a fault they don't need to be redone and the algorithm can restart quicker. Otherwise, one can let algorithms adaptively run on a dynamic number of cores, so that in case of a fault the algorithm can continue running after a unit of the computer is shut off. This would include ongoing error measurements and decisions on whether possibly corrupted data should be excluded combined with modular composition of the simulation.
        \null\newpage
        
        \section*{Exercise 8}
        \subsection*{(a)}
The Mandates are as follows:
\begin{enumerate}
\item Better resolve a model's full, natural range of length or time scales
\item Accommodate physical effects with greater fidelity
\item Allow the model degrees of freedom in all relevant dimensions
\item Better isolate artificial boundary conditions
\item Combine multiple complex models
\item Solve an inverse problem or perform data assimilation
\item Perform otimization or control
\item Quantify uncertainty
\item Accomplish predictions without physical models using statistical models based on large data sets
\end{enumerate}

\subsection*{(b)}
Accomodating physical effects with greater fidelity is an important opportunity in aerodynamic modeling for e.g. aircraft design. This includes the modeling of turbulent flows and shock waves. Still, many costly experiments need to be carried out in the wind tunnel including people and material as their virtual numerical pendants in computational fluid dynamics are not feasible. This includes especially numerical simulations of high-Reynolds-number flows. Also, wind tunnel experiments are physically not optimal to see all possible effects on a moving airplane as even here many approximations need to be made to translate the results into a real life situation. One of the potentials of HPC in this area is that given enough computation power, simulations might become more accurate than wind tunnel experiments. This raises a second important point as it will not only safe costs in the production, but it will make airplanes safer and more efficient in practice.
        \null\newpage
        
\end{document}














